
\documentclass[11pt,letterpaper]{article}
\usepackage{amsmath,amsfonts}
\usepackage{hyperref}
\setlength{\oddsidemargin}{0in}
\setlength{\evensidemargin}{0in}
\setlength{\textwidth}{6.5in}
\setlength{\topmargin}{-0.5in}
\setlength{\textheight}{8.7in}
\setlength{\parskip}{0.35em}
\parindent0em
\usepackage{graphicx}
\usepackage{amsthm}
\author{Alexander Cai, Oliver Cheng}
\date{\today}
\title{Multi Agent Meta Reinforcement Learning on Neural Networks}

\newcommand*{\vertbar}{\rule[-1ex]{0.5pt}{2.5ex}}
\newcommand*{\horzbar}{\rule[.5ex]{2.5ex}{0.5pt}}

\newenvironment{soln}{\begin{proof}[\bf{Solution}]}{\end{proof}\hrule}
\newcommand\restr[2]{{% we make the whole thing an ordinary symbol
  \left.\kern-\nulldelimiterspace % automatically resize the bar with \right
  #1 % the function
  \vphantom{\big|} % pretend it's a little taller at normal size
  \right|_{#2} % this is the delimiter
  }}
\usepackage{parskip}


\usepackage{cite}

\begin{document}
\maketitle
\section{Introduction to Meta-Reinforcement Learning}
In recent years there has been a growing amount of excitement about 
\textit{meta-learning} in order to solve a wider set of problems. This is 
often used in reinforcement learning tasks. In a standard reinforcement
 learning task, we have a policy that governs how an agent transitions 
between states in an environment. There are rewards that the agent can 
receive, and the optimal policy is one that maximizes rewards. We can 
formalize this by defining a reinforcement learning regime as a Markov 
Decision Process (MDP) where $S$ is the set of states, $A$ is the set 
of actions, $T$ is the transition distribution (how states transition given 
an action), and $R$ is the reward, which is a function of the state. Let 
$\pi_\theta$ be the policy, which is parameterized by some $\theta$. 
Thus for some "task" $\mathcal{T}$, the goal is thus to maximize the objective 
\[ \mathcal{J}_{\mathcal{T}}(\theta) = \mathbb{E}_{a_t \sim \pi_\theta(s_t), s_{t+1}
 \sim T(s_t, a_t)} \left( \sum_t \gamma^t R(s_t, a_{t-1}) \right).\]
The task $\mathcal{T}$ determines the rewards, and hence the policy that is optimized 
depend heavily on what the task $\mathcal{T}$ is. 

Reinforcement learning is used to solve many problems such as Go, Chess, 
Atari games, etc. where the environment is unknown and is structured in 
a way which an agent must discover an optimal strategy. This framework 
gives us a way to view human beings and the nervous system as a 
reinforcement learning program where we humans are the agent. 
(Un)Fortunately, this is a bit too basic of a way of modeling human 
beings and the brain, as reinforcement learning networks have not taken 
over humanity yet (citation needed). One of the reasons is that rewards 
and the task that an agent needs to optimize a policy for is constantly 
changing in the real world. Thus we introduce the idea of meta-learning, 
where rather than optimizing a single policy for a single task $\mathcal{T}$, 
we optimize for a learning strategy on how to optimize a policy for a distribution 
of tasks $p(\mathcal{T})$. This idea is motivated by the idea in psychology of 
 "learning how to learn" and is conceptually closer to how the brain 
works (that one prefrontal cortex Deepind paper.). With meta-learning, due to
the way we are improving the way we are finding the policy, this is a way
we can introduce the idea of "past experiences" to help train policy for 
tasks that are similar.

\section{Meta-Reinforcement Learning in the Brain}

Basically try and repeat the thing in Box 4 of https://www.sciencedirect.com/science/article/pii/S1364661319300610




maybe this is garbage lol

In the Theory of Neural Computation, we can see reinforcement-learning-esque
regimes. In particular researchers have found that the dopamine system seems
to follow the TD algorithm, a reinforcement learning algorithm. In particular, the error
function of the TD algorithm operates in the same way to how dopamine neurons learn their firing rates in order to 
associate stimuli with future rewards. In this case, the dopamine neuron is the agent
and it can change its firing rate as an action. 





\section{Implementation of Meta-Learning MARL}



\section{Comparison to Known Learning Rules}

We compare this meta-learning (learning of the best learning rule) to accepted learning rules in the literature. Stochastic gradient descent (SGD) converges the quickest, but due to the weight transport problem, is not biologically plausible (citation needed). Biologically plausible alternatives involve learning from a global error signal, and involve perturbation-type learning rules. Global error signals have been observed in the brain, but it is still under much research exactly how the brain uses these error signals in order to learn and update neurons. We will look at node (NP) and weight perturbation (WP).

The idea behind node or weight perturbation is to take some perturbation $\*\xi$ of either
the weights or the nodes, and then see how that changes the objective or loss. The learning rule update
then corresponds to scaling this perturbation by whether or not it increases or decreases
the loss. Consider a general neural network with $L$ layers and a data set $\{\*x_i^{(0)},\*y_i\}_{i=1}^N$. 
Let $\*W_{\ell} \in \R^{R_{\ell+1} \times R_{\ell}}$ where $R_\ell$ is the number of 
neurons in layer $\ell$. We define the forward pass as
\begin{equation}\label{eqn1}\*X^{(\ell)} = \sigma\left(\*W_{\ell-1} \*X^{(\ell-1)\top}\right).\end{equation}
Where $\sigma$ is some non-linearity such as ReLU. We use MSE loss 
\[E = \frac 12 \| \*X^{(L)}- \*Y\|^2.\]
In SGD, we use backpropagation to update the weight parameters by 
\[\Delta_{SGD} \* W_{\ell} = \eta\frac{\partial E}{\partial \*W_{\ell}},\] 
where the chain rule means we 
need to "transport" these values backwards through the neural network. On the other 
hand, in WP/NP, we have two forward passes. For the first, we define the same as \ref{eqn1}, 
however for NP, we will let the perturbation be a vector $\xi \in \R^{R_{\ell}}$ where all entries are distributed 
i.i.d $\mathcal{N}(0,\sigma)$. We can then define the forward pass as 
\[ \widetilde{\*X}^{(\ell)} = \sigma\left(\*W_{\ell -1} \widetilde{\*X}^{(\ell-1)\top} + \*\xi^{(\ell)} \right).\]
While for WP we take $\*\Xi^{(\ell)} \in \R^{R_{\ell + 1 }\times R_\ell}$ where all entries are distributed 
i.i.d $\mathcal{N}(0,\sigma)$. 
\[ \hat{\*X}^{(\ell)} = \sigma\left(\left(\*W_{\ell -1} + \*\Xi^{(\ell-1)}\right) \hat{\*X}^{(\ell-1)\top}  \right).\]
We can then define two errors for each NP and WP respectively.
\[ E_N = \frac 12 \| \widetilde{\* X}^{(L)} - \* Y \|^2 \qquad\qquad E_W = \frac 12 \| \hat{\* X}^{(L)} - \* Y \|^2. \]
From here we can simply transmit the global error signal $E-E_N$ or $E-E_W$ and update our weights according
to this error signal. In particular, we update parameters as follows for NP
\[ \Delta_{NP} \*W_{\ell} =\frac{\eta}{\sigma} (E-E_N) \sum_{i=1}^{R_{\ell+1}} \*x^{(\ell)}_i \* \xi^{(\ell+1)\top} \]
While for WP we update as
\[ \Delta_{WP} \*W_{\ell} = \frac{\eta}{\sigma} (E-E_W) \* \Xi^{(\ell)} \]
 The idea is that perturbations which cause decrease the objective will be added to 
 the parameters of the model. In expectation we have that up to a constant
\[ \left\langle \Delta_{NP}\*W_{\ell} \right\rangle_{\*\xi^{(\ell)}}= \left\langle \Delta_{WP}\*W_{\ell} \right\rangle_{\*\Xi^{(\ell)}} =  \Delta_{SGD}\*W_{\ell}.\]

 
 
Node and weight perturbation are often used in small teacher-student learning cases 
where there is only a single perceptron. Recent work by Hiratani et al. explores node perturbation
in neural networks with a hidden layer, and finds computational and stability issues that
suggest biological implausibility (cite hiratani et al). Nonetheless, these perturbation techniques remain an interesting way
to investigate possible learning rules for neurons. 




\end{document}
